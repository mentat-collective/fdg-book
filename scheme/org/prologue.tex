% Created 2022-05-02 Mon 08:28
% Intended LaTeX compiler: pdflatex
\documentclass[11pt]{article}
\usepackage[utf8]{inputenc}
\usepackage[T1]{fontenc}
\usepackage{graphicx}
\usepackage{grffile}
\usepackage{longtable}
\usepackage{wrapfig}
\usepackage{rotating}
\usepackage[normalem]{ulem}
\usepackage{amsmath}
\usepackage{textcomp}
\usepackage{amssymb}
\usepackage{capt-of}
\usepackage{hyperref}
\usepackage{minted}
\author{Sam Ritchie}
\date{\today}
\title{}
\hypersetup{
 pdfauthor={Sam Ritchie},
 pdftitle={},
 pdfkeywords={},
 pdfsubject={},
 pdfcreator={Emacs 27.2 (Org mode 9.5.2)},
 pdflang={English}}
\begin{document}

\tableofcontents


\section{Prologue}
\label{sec:org12af31c}

\subsection{Programming and Understanding}
\label{sec:orgd94f6e3}

One way to become aware of the precision required to unambiguously communicate a
mathematical idea is to program it for a computer. Rather than using canned
programs purely as an aid to visualization or numerical computation, we use
computer programming in a functional style to encourage clear thinking.
Programming forces us to be precise and unambiguous, without forcing us to be
excessively rigorous. The computer does not tolerate vague descriptions or
incomplete constructions. Thus the act of programming makes us keenly aware of
our errors of reasoning or unsupported conclusions.\footnote{The idea of using computer programming to develop skills of clear
thinking was originally advocated by Seymour Papert. An extensive discussion of
this idea, applied to the education of young children, can be found in Papert
[13].}

Although this book is about differential geometry, we can show how thinking
about programming can help in understanding in a more elementary context. The
traditional use of Leibniz’s notation and Newton’s notation is convenient in
simple situations, but in more complicated situations it can be a serious
handicap to clear reasoning.

A mechanical system is described by a Lagrangian function of the system state
(time, coordinates, and velocities). A motion of the system is described by a
path that gives the coordinates for each moment of time. A path is allowed if
and only if it satisfies the Lagrange equations. Traditionally, the Lagrange
equations are written $${\frac{d}{dt}\frac{\partial L}{\partial \dot{q}}} -
\frac{\partial L}{\partial q}=0.$$ What could this expression possibly mean?

Let’s try to write a program that implements Lagrange equations. What are
Lagrange equations for? Our program must take a proposed path and give a result
that allows us to decide if the path is allowed. This is already a problem; the
equation shown above does not have a slot for a path to be tested.

So we have to figure out how to insert the path to be tested. The partial
derivatives do not depend on the path; they are derivatives of the Lagrangian
function and thus they are functions with the same arguments as the Lagrangian.
But the time derivative \(d/dt\) makes sense only for a function of time. Thus we
must be intending to substitute the path (a function of time) and its derivative
(also a function of time) into the coordinate and velocity arguments of the
partial derivative functions.

So probably we meant something like the following (assume that \(\omega\) is a
path through the coordinate configuration space, and so \(w(t)\) specifies the
configuration coordinates at time \(t\)):

$$\frac{d}{d t}\left( \left.\frac{\partial L(t, q, \dot{q})}{\partial \dot{q}}
\right|_{\substack{ {q=w(t)} \\ {\dot{q}=\frac{d w(t)}{d t}} }}
\right)-\left.\frac{\partial L(t, q, \dot{q})}{\partial q}\right|_{ \substack{
q=w(t) \\ {\dot{q}=\frac{d w(t)}{d t}}} }=0.$$

In this equation we see that the partial derivatives of the Lagrangian function
are taken, then the path and its derivative are substituted for the position and
velocity arguments of the Lagrangian, resulting in an expression in terms of the
time.

This equation is complete. It has meaning independent of the context and there
is nothing left to the imagination. The earlier equations require the reader to
fill in lots of detail that is implicit in the context. They do not have a clear
meaning independent of the context.

By thinking computationally we have reformulated the Lagrange equations into a
form that is explicit enough to specify a computation. We could convert it into
a program for any symbolic manipulation program because it tells us \emph{how} to
manipulate expressions to compute the residuals of Lagrange’s equations for a
purported solution path.\footnote{The \emph{residuals} of equations are the expressions whose value must be zero
if the equations are satisfied. For example, if we know that for an unknown \(x\),
\(x^3-x=0\) then the residual is \(x^3 - x\). We can try \(x = -1\) and find a
residual of 0, indicating that our purported solution satisfies the equation. A
residual may provide information. For example, if we have the differential
equation \(df(x)/dx - af(x) = 0\) and we plug in a test solution \(f(x) = Ae^{bx}\)
we obtain the residual \((b-a)Ae^{bx}\), which can be zero only if \(b = a\).}

\subsection{Functional Abstraction}
\label{sec:org3927372}

But this corrected use of Leibniz notation is ugly. We had to introduce
extraneous symbols (\(q\) and \(\dot{q}\)) in order to indicate the argument
position specifying the partial derivative. Nothing would change here if we
replaced \(q\) and \(\dot{q}\) by \(a\) and \(b\).\footnote{That the symbols \(q\) and \(\dot{q}\) can be replaced by other arbitrarily
chosen nonconflicting symbols without changing the meaning of the expression
tells us that the partial derivative symbol is a logical quantifier, like forall
and exists (\(\forall\) and \(\exists\)).} We can simplify the notation by
admitting that the partial derivatives of the Lagrangian are themselves new
functions, and by specifying the particular partial derivative by the position
of the argument that is varied

$$\frac{d}{d t}\left(\left(\partial_{2} L\right)\left(t, w(t), \frac{d}{d t}
w(t)\right)\right)-\left(\partial_{1} L\right)\left(t, w(t), \frac{d}{d t}
w(t)\right)=0,$$

where \(\partial_{i}L\) is the function which is the partial derivative of the
function L with respect to the ith argument.\footnote{The argument positions of the Lagrangian are indicated by indices
starting with zero for the time argument.}

Two different notions of derivative appear in this expression. The functions
\(\partial_2 L\) \(\partial_1 L\), constructed from the Lagrangian \(L\), have the
same arguments as \(L\).

The derivative \(d/dt\) is an expression derivative. It applies to an expression
that involves the variable \(t\) and it gives the rate of change of the value of
the expression as the value of the variable \(t\) is varied.

These are both useful interpretations of the idea of a derivative. But functions
give us more power. There are many equivalent ways to write expressions that
compute the same value. For example \(1/(1/r_1 + 1/r_2)=(r_1r_2)/(r_1 + r_2)\).
These expressions compute the same function of the two variables \(r_1\) and
\(r_2\). The first expression fails if \(r_1 = 0\) but the second one gives the
right value of the function. If we abstract the function, say as \(\Pi(r_1,
r_2)\), we can ignore the details of how it is computed. The ideas become clearer
because they do not depend on the detailed shape of the expressions.

So let’s get rid of the expression derivative \(d/dt\) and replace it with an
appropriate functional derivative. If \(f\) is a function then we will write \(Df\)
as the new function that is the derivative of \(f\):\footnote{An explanation of functional derivatives is in Appendix B, page 202.}

$$(D f)(t)=\left.\frac{d}{d x} f(x)\right|_{x=t}.$$

To do this for the Lagrange equation we need to construct a function to take the
derivative of.

Given a configuration-space path \(w\), there is a standard way to make the
state-space path. We can abstract this method as a mathematical function
\(\Gamma\):

$$\Gamma[w](t)=\left(t, w(t), \frac{d}{d t} w(t)\right).$$

Using \(\Gamma\) we can write:

$$\frac{d}{dt}\left(\left(\partial_{2} L\right) \left(\Gamma[w](t)\right)
\right)\right) - \left(\partial_{1} L\right) \left(\Gamma[w](t)\right)=0.$$

If we now define composition of functions \((f \circ g)(x) = f(g(x))\), we can
express the Lagrange equations entirely in terms of functions:

$$D\left(\left(\partial_{2} L\right) \circ \left(\Gamma[w]\right)\right)
\\ -\left(\partial_{1} L\right) \circ \left(\Gamma[w]\right)=0.$$

The functions \(\partial_1 L\) and \(\partial_2 L\) are partial derivatives of the
function \(L\). Composition with \(\Gamma[w]\) evaluates these partials with
coordinates and velocites appropriate for the path \(w\), making functions of
time. Applying \(D\) takes the time derivative. The Lagrange equation states that
the difference of the resulting functions of time must be zero. This statement
of the Lagrange equation is complete, unambiguous, and functional. It is not
encumbered with the particular choices made in expressing the Lagrangian. For
example, it doesn’t matter if the time is named \(t\) or \(\tau\), and it has an
explicit place for the path to be tested.

This expression is equivalent to a computer program:\footnote{The programs in this book are written in Scheme, a dialect of Lisp. The
details of the language are not germane to the points being made. What is
important is that it is mechanically interpretable, and thus unambiguous. In
this book we require that the mathematical expressions be explicit enough that
they can be expressed as computer programs. Scheme is chosen because it is easy
to write programs that manipulate representations of mathematical functions. An
informal description of Scheme can be found in Appendix A. The use of Scheme to
represent mathematical objects can be found in Appendix B. A formal description
of Scheme can be obtained in [10]. You can get the software from [21].}

\begin{minted}[]{scheme}
(define ((Lagrange-equations Lagrangian) w)
  (- (D (compose ((partial 2) Lagrangian) (Gamma w)))
     (compose ((partial 1) Lagrangian) (Gamma w))))
\end{minted}

In the Lagrange equations procedure the parameter \texttt{Lagrangian} is a procedure
that implements the Lagrangian. The derivatives of the Lagrangian, for example
\texttt{((partial 2) Lagrangian)}, are also procedures. The state-space path procedure
\texttt{(Gamma w)} is constructed from the configuration-space path procedure \texttt{w} by
the procedure \texttt{Gamma}:

\begin{minted}[]{scheme}
(define ((Gamma w) t)
  (up t (w t) ((D w) t)))
\end{minted}

where \texttt{up} is a constructor for a data structure that represents a state of the
dynamical system (time, coordinates, velocities).

The result of applying the \texttt{Lagrange-equations} procedure to a procedure
\texttt{Lagrangian} that implements a Lagrangian function is a procedure that takes a
configuration-space path procedure \texttt{w} and returns a procedure that gives the
residual of the Lagrange equations for that path at a time.

For example, consider the harmonic oscillator, with Lagrangian

$$L(t, q, v) = \frac{1}{2}mv^2 - \frac{1}{2}kq^2,$$

for mass \(m\) and spring constant \(k\). this lagrangian is implemented by

\begin{minted}[]{scheme}
(define ((L-harmonic m k) local)
  (let ((q (coordinate local))
        (v (velocity local)))
    (- (* 1/2 m (square v))
       (* 1/2 k (square q)))))
\end{minted}

We know that the motion of a harmonic oscillator is a sinusoid with a given
amplitude \(a\), frequency \(\omega\), and phase \(\varphi\):

$$x(t) = a \cos(\omega t + \varphi).$$

Suppose we have forgotten how the constants in the solution relate to the
physical parameters of the oscillator. Let’s plug in the proposed solution and
look at the residual:

\begin{minted}[]{scheme}
(define (proposed-solution t)
  (* 'a (cos (+ (* 'omega t) 'phi))))

(show-expression
 (((Lagrange-equations (L-harmonic 'm 'k))
   proposed-solution)
  't))

;; should produce \cos(\omega t + \varphi) a (k-m\omega^2)
\end{minted}

The residual here shows that for nonzero amplitude, the only solutions allowed
are ones where \((k - m\omega^2) = 0\) or \(\omega = \sqrt{k/m}\).

But, suppose we had no idea what the solution looks like. We could propose a
literal function for the path:

\begin{minted}[]{scheme}
(show-expression
 (((Lagrange-equations (L-harmonic 'm 'k))
   (literal-function 'x))
  't))
;; should produce $$kx(t)+mD^2 x(t)$$
\end{minted}

If this residual is zero we have the Lagrange equation for the harmonic
oscillator.

Note that we can flexibly manipulate representations of mathematical functions.
(See Appendices A and B.)

We started out thinking that the original statement of Lagrange’s equations
accurately captured the idea. But we really don’t know until we try to teach it
to a naive student. If the student is sufficiently ignorant, but is willing to
ask questions, we are led to clarify the equations in the way that we did. There
is no dumber but more insistent student than a computer. A computer will
absolutely refuse to accept a partial statement, with missing parameters or a
type error. In fact, the original statement of Lagrange’s equations contained an
obvious type error: the Lagrangian is a function of multiple variables, but the
\(d/dt\) is applicable only to functions of one variable.
\end{document}